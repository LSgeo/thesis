\documentclass{article}
\usepackage{amsmath}
\usepackage[style=authoryear]{biblatex}
\usepackage{graphicx}
\usepackage[]{hyperref}
\usepackage{tabularray}
\usepackage{siunitx}
\usepackage[noabbrev]{cleveref}
\addbibresource{bib/PhD.bib}
\begin{document}

\author{\and{Luke Thomas Smith} \and{Tom Horrocks} \and {Eun-Jung Holden} \and{Daniel Wedge} \and{Naveed Akhtar}}
\title{Implicit Neural Representation Case Studies for Potential Field Geophysics}
\date{\today}
\maketitle{}

\begin{abstract}

\end{abstract}

\section{Introduction}
    The advent of periodic activation functions for the well known multilayer perceptron (or fully connected network) has created a new research field, known as Implicit Neural Representation.
    Implicit Neural Representation (INR) is the task of representing a signal as a function of its coordinates.
    Such functions are known as coordinate multilayer perceptrons, or Coordinate MLPs.
    By parameterising the function in coordinate space, novel values of the function can be queried at coordinates within the continuous spatial domain.
    Periodic activation functions differ from standard linear activations, such as ReLU, by comprising the neuron with a function that repeats periodically.
    Such functions include sinusoids \parencite{sitzmann2019siren}, Gaussian \parencite{ramasinghePeriodicityUnifyingFramework2022}, and most recently, wavelets \parencite{saragadamWIREWaveletImplicit2023}.

    Geophysical potential fields are continuous functions in three dimensions, and are of importance to mineral exploration and geoscience.
    When surveying these fields for exploration geophysics, sample locations are scattered in x, y, and z, despite best efforts to conform to a regular grid or line sample regime.
    Regularising these samples to a two dimensional grid for interpretation, storage, and further machine processing is a well studied process with many methods available.
    After regularisation, the grid arrays can be subject to a large number of widely used filters, such as horizontal and vertical gradient calculation and their derivatives.
    These filters operate on the grid product, a representation of the potential field quantised by the cell spacing defined during the regularisation process.

    If a potential field survey could instead be encoded within an Implicit Neural Network as an INR, filters could be calculated analytically in the continuous three dimensional domain.
    With this representation the extraction of regularised two dimensional slices, namely grids, could be performed \emph{a posteri} of gridding.
    This would allow the use of all available original sample data in calculation of filters, and reduce the manual intervention required during the gridding process.
    Additionally, due to the continuous representation, grid resolution is not constrained by cell size selection.
    



\subsection{Implicit Neural Representation}
\label{sec:inr}
Realising their power in implicit representation for natural images, many works quicky investigated 
\begin{enumerate}
    \item MLPs
    \item Periodic activations
    \item Storage weights not cells
\end{enumerate}

\subsection{Geophysical Sampling}
\label{sec:geo_sampling}
Here, I discuss sampling in geophysics 

\section{Method}
\subsection{SIREN}
\begin{enumerate}
    \item SIREN (or WIRE, if it improves)
    \item Regularisation loss
    \item Parameter opt
\end{enumerate}

\subsection{Training Particulars}
Performed on an Intel i5-12400 CPU with 32 GB RAM, and an Nvidia 3060 12 GB GPU. 

ADAM, One Cycle LR, 1000 iter/epochs.

MSE loss.

\subsection{Data}
GADDS Wolfe Creek impact survey netCDF

Noddyverse

tba

\section{Results}
\subsection{Implicit gridding}
\begin{enumerate}
    \item Best attempt at recreating grid as gridded by GA.
    \item Demonstration of the range of altitude slices
\end{enumerate}

\subsection{Spatial gradients and filters}
\begin{enumerate}
    \item Horizontal Gradients
    \item Vertical Gradients
    \item Combinations and filters
\end{enumerate}

\section{Discussion}
\begin{enumerate}
    \item Limitations: Capacity of Network
    \item RAM usage (if I can't do batched C-MLP training)
    \item Accuracy of learnt x,y and z gradients
\end{enumerate}

\section{Conclusions}

\section{Acknowledgements}
This project is funded by a Rio Tinto Iron Ore PhD scholarship.

\printbibliography{}

\end{document}