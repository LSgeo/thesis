\documentclass[manuscript.tex]{subfiles}
\begin{document}

\section{Discussion} %brings together the key findings and contributions of the research as a whole
% provide a synthesis of the work as a whole presented in the body of the thesis.
% It is important that this final chapter demonstrates the collective contribution to knowledge provided by the research described in the thesis and provide overall conclusions that address the research aims.


One of the challenges identified in \cref{paper1} was the need for large extents of suitable training data.
During the initiation of \cref{paper2}, a large amount of geologically varied synthetic training data was released as \emph{the Noddyverse} \parencite{jessellNoddyverseMassiveData2022}.
The method for creating synthetically sampled line profiles form potential field grids, like those available in the Noddyverse, can be extended and used to train the networks described in \cref{paper1}



Throughout this thesis, methods developed in computer vision research for natural images have been extended to geophysical grid rasters.
Despite the success of the contributed methods, an important question has remained unaddressed.
Do grid rasters in potential field geophysics share the same properties as natural images?
More precisely, \emph{Are potential field grids natural signals?}
If so, it can be expected that future methods in deep learning developed for natural images will be readily extensible to Geophysics.
\Textcite{donohoCompressedSensing2006}

\subsection{Are potential field grids natural signals?}
% Questions:
% Are geophysical grids natural signals?
% -   Lipschitz smoothness
% -   Fourier power spectra fall-off
% -   There's a citation in Compressed sensing 2006 about natural signals, from Donohan or something.



\end{document}