\documentclass[manuscript.tex]{subfiles}
\begin{document}
\setcounter{chapter}{1}


% indicates that an original and coherent argument is being developed and that the research is making a contribution to existing literature
% outline the topic, justify the research, identify research aims and outline thesis structure

% Ensure acronyms correctly used

\section{Potential field geophysics in exploration geoscience}
\label{sec:introgeo}
%- what are they used for (summary) 
The subsurface continuation of exposed geology can be extrapolated and interpolated using structures mapped in surveys of the Earth's naturally occuring potential fields.
Mapping methods using these fields are routinely used in geoscience for low-cost investigations of the subsurface over a large spatial volume.
For this reason, magnetic field surveys are commonly used by geophysicists for geological mapping and interpretation \parencite{nabighian75thAnniversaryHistorical2005}.
In addition, potential field data are used as inputs to computer processing tasks such as numerical modelling and inversion.
In these tasks, various dimensions of geoscience data are integrated to make predictions regarding the subsurface.
These and other geophysical methods are becoming more important to exploration as resources are increasingly sought from below sedimentary cover.

\subsection{Aeromagnetic surveys}
% how are they gathered (surveys)
One of the fundamental methods routinely used for mineral exploration are aerogeophysical surveys of the Earth's magnetic or gravitational potential fields.
Airborne surveys are acquired at a range of scales at different stages of exploration, from pre-competitive regional data to stimulate exploration \parencite{howardAirborneGeophysicalCoverage2004}, to targeted investigations by the exploration industry at the prospect scale of up to several square kilometres.
Surveys are also routinely collected on the ground, or less commonly in exploration, by ship or satellite, with most processing methods and interpretation shared between all forms of acquisition.
Aeromagnetic surveys are among the lowest cost geophysical method \parencite{dentithGeophysicsMineralExploration2014}, however the cost rapidly increases when surveying at higher resolution due to overheads in flight line sampling.
When collecting data, aircraft fly straight transect lines at a nominal safe elevation above topography, known as the topography drape.
Surveys are flown with a nominal heading for the direction that data will be acquired along line.
For regional surveys this is recommended to be North-South, however early practice was perpendicular to the dominant geological strike, under the expectation of sampling the greatest frequency of change in magnetic intensity \parencite{islesGeologicalInterpretationAeromagnetic2018}.
Because the power in higher frequencies diminishes with increasing source-sensor distance, the lower the height, the more high-frequency components will be captured by the sensor, a fact which will become important in subsequent sections.
Acquisition along flight lines is sampled at a rate of \qty{10}{\hertz} or greater, which corresponds to an interval of \qty{7}{\m} or shorter at nominal flight speeds of \qty{70}{\m\per\s} \parencite{goodwinAirborneMagneticRadiometric2023}.
The spacing between lines is a factor of the scale of investigation and cost, and regional pre-competitive data in Australia targets state-wide coverage at \qty{400}{\m}, which is generally regarded as the upper limit of usefulness to exploration \parencite{howardAirborneGeophysicalCoverage2004}.
These low-resolution surveys, funded by government geoscience agencies, are a key driver in greenfields exploration and are released as open file data.
Higher-resolution surveys over prospective areas fly lines at closer line spacings, of the order \qty{100}{\m}, in order to resolve greater detail.
The importance of these parameters in relation to resolvable detail will be outlined shortly.
The collected data may contain millions of samples scattered in three dimensions, and interpretation or processing often relies on regularisation to a quantised grid.

\subsection{Survey data regularisation}
% - how are they transformed (gridding)
Survey data can be interpreted as one dimensional transects, but more commonly they are regularised to a two-dimensional grid raster for interpretation in a process termed gridding.
The resulting interpolated array contains uniformly spaced nodes and is referred to as a grid.
Grids are raster data, where each pixel stores a scalar measurement of a specific property.
In geophysics and throughout this manuscript, these are referred to as cells, and the property is a measure of the strength of the magnetic field at a specific location.
The field strength measured is the Total Magnetic Intensity (TMI), which is the vector sum of each directional vector of the potential \parencite{blakelyPotentialTheoryGravity1996}.
This scalar value is recorded in nanoTesla and may be negative with an unbounded range that can exceed \qty{50,000}{\nano\tesla} in the Earth's environment.
The location recorded at a sample point is the geographic latitude and longitude, and altitude, acquired with high-precision differential GPS, alongside radar altimeter elevation.
These data may be transformed to a flat projected system, where spatial dimensions are expressed in meters, used throughout this manuscript.
There are numerous well-established gridding methods, and novel methods are frequently proposed in the literature.
The simplest is known as nearest neighbour interpolation, where each node is set to equal to the closest sample value.
Of those most commonly used in potential field geophysics, these are minimum curvature \parencite{briggsMachineContouringUsing1974}, splines \parencite{bhattacharyyaBicubicSplineInterpolation1969,shureHarmonicSplinesGeomagnetic1982,smithGriddingContinuousCurvature1990}, and equivalent sources \parencite{dampneyEquivalentSourceTechnique1969, solerBetterStrategyInterpolating2020}.
Splines are used in bi-directional gridding, which is well-suited for aeromagnetic surveys, where the sample data are first interpolated in the line parallel direction, and the intermediate product is interpolated in the line perpendicular direction \parencite{dentithGeophysicsMineralExploration2014}.
In other areas of geoscience, a statistical interpolator is frequently used, known as kriging \parencite{hansenInterpretiveGriddingAnisotropic1993,davis1986statistics}.
Some recently proposed gridding methods include the methods of \textcite{naprstekNewMethodInterpolating2019}, \textcite{xuGravityAnomalyReconstruction2019}, or \textcite{chenPotentialFieldData2022}.
Each interpolation method attempts to predict regularly spaced grids by leveraging properties intrinsic to the data, with the possible explicit integration of \emph{a priori} knowledge such as geophysical laws or geological context.
A review of spatial interpolation methods is provided in \textcite{liReviewComparativeStudies2011}

\subsection{A definition for geophysical resolution}
In all interpolated grids, the spatial dimension of each cell is referred to as the cell size, with the unit of metre used in this manuscript.
As stated above, resolvable detail in geophysical surveys is a product of the distance between the causative magnetic body and sensor (height), and the line spacing \parencite{islesRelationshipsGeologicalResolution1992,islesGeologicalInterpretationAeromagnetic2018,dentithGeophysicsMineralExploration2014}.
Height is easily controlled by flying the lowest safe and sufficiently smooth topography drape, and along-line sampling is performed at a constant high rate, so the key factor controlling resolution in aeromagnetic survey design is the line spacing.
Further survey design considerations for airborne geophysics are outlined in \textcite{goodwinAirborneMagneticRadiometric2023,islesGeologicalInterpretationAeromagnetic2018,reidAeromagneticSurveyDesign1980}.
Each gridding method includes a final cell size parameter selectable by the geophysicist, with a long-standing guideline of one third to one fifth the line spacing.
This sufficiently preserves detail in areas of high sampling, while avoiding excessive interpolated cells in areas of low sampling, which may lead to artefacts.
Because of the sampling frequency of nominally \qty{7}{\m} in the flight line direction, and  up to \qty{400}{\m} in the line perpendicular direction, the resulting grid resolution will be anisotropic at best, and uniformly low-resolution at worst.
An important property of sampled data is the Nyquist frequency, \[f_{Nyquist} = f_{sampling} / 2.\]
This is the lowest frequency at which a bandwidth limited signal may be sufficiently sampled, beyond which higher frequencies will be incorrectly recorded as aliasing.
An aliased signal contains spurious low-frequency features caused by the sampling, not the underlying signal.
This limiting frequency applies to each sampling direction in surveys, as well as the cell size of the grid.
The smallest resolvable feature in each of these can be spatially described by wavelength, as the inverse of frequency.
In low-resolution aeromagnetic surveys of \qty{400}{\m} line spacing, strong variations of magnetic intensity in the line-perpendicular direction  that occur over a distance shorter than \qty{800}{\m} will be aliased.
When gridded at \qty{80}{\m} cell size, the corresponding limit on resolvable detail is features as small as \qty{160}{\m}.
% This choice of cell size will effectively filter out features in the line-parallel direction that vary strongly, from those possibly present at \qty{14}{\m} through to \qty{160}{\m}.
Depending on the complexity of the geological domain, this may cause spatial aliasing in one direction.
This is visible in grids where features take on a blocky or stepped appearance, referred to as a beading artefact, or generally in this manuscript as a low-resolution artefact.
Low-resolution gridded aeromagnetic data refers to surveys with a wide line spacing causing a lack of high-frequency details in the line-perpendicular direction, resulting in potential artefacts or missing features in grid data.

Here, we arrive at the definition of resolution adopted throughout this thesis.

\emph{Resolution in potential field geophysics refers to the sampling of high-frequencies present in the potential field caused by a geological source.
    Resolution is further defined by the preservation of these frequencies in gridded potential field data.}

% - How are they used (geo interp, modelling, filtering) (detailed)
% levelling
% geological interpretation
% modellin
%

\section{Geophysics data in machine learning}
\label{sec:introdata}
% e.g. limited training data to deal with diverse geological features, geology various regionally - different potential fields features and range, etc....
Natural image raster data are ubiquitous in computer vision machine learning.
These are most commonly three channel arrays containing normalised brightness values in the red, green and blue channels.
Resolution in these rasters is typically a product of an imaging system or software defined grid with regular sampling.
Like in geophysics, these raster data have an array height and width, which is referred to as resolution.
However, the lack of spatial quantification for the size of natural images confuses the term with its definition for geophysical resolution.

Resulting from the processes described in \Cref{sec:introgeo}, geophysical rasters are highly distinct from the image data used widely in computer vision.
This includes the number of channels (3 in natural images, 1 in aeromagnetics), range (255 compared to 50,000 or more), and anisotropy of resolution.
Additionally, the features present within these data are highly dissimilar.
While natural images may contain common objects which share high-level features such as eyes, faces, or complex geometry, these data only share low-level features with aeromagnetic data, such as simple line and curve components.
This fact limits the applicability of transfer learning from methods in computer vision machine learning \parencite{tanSurveyDeepTransfer2018} to geoscience data.
These points also raise the uncertainty of methods developed for natural images to be  applicable to geoscience data.

Geoscience data faces a number of challenges distinct from natural images. These have been outlined in \textcite{karpatneMachineLearningGeosciences2019}, and relevant to this work are:

\begin{itemize}
    \item{} Spatio-Temporal Structure: Aeromagnetic grids are spatially corellated, neighbouring cells contain similar values arising from the same causative volume.
    \item{} Multi-Source Multi-Resolution Data: One location may be covered by mutiple aeromagnetic surveys with different resolutions, or the same survey may be presented with different levels of processing.
    \item{} Poor Quality of Data: Sample data are subject to numerous sources of noise, which combine to become a significant fraction of the gridded data value, or be present as aliased frequencies.
    \item{} Small Sample Size: While total aeromagnetic coverage in Australia is high, the high-resolution extents are spatially limited and biased by geological domain.
\end{itemize}

One approach to overcoming the challenges faced in real-world geoscience data is to use synthetic data.
During this thesis, a large collection of synthetic data became available, containing realistic 3D geological voxel models and 2D geophysical forward models \parencite{jessellNoddyverseMassiveData2022}.
These data present a large amount of uniformly sampled and noise free grids from an unbiased set of geological domains.


\section{Traditional methods for enhancing resolution}
As outlined in \Cref{sec:introgeo}, resolution is a product of sample distance and density, and gridded rasters have a specific number of cells in each axis depending on the spatial extent.
Upsampling raster data to larger height and width dimensions is commonplace, and generally performed by numerical image filters such as the cubic spline \parencite{keysCubicConvolutionInterpolation1981} or Lanczos filter \parencite{lanczos1988applied}.
Some of these interpolation methods are the basis of the gridding methods previously outlined, and simply selecting a smaller cell size during gridding is sufficient to achieve a larger image raster.
These filters are adept at imputing the required cell values while preserving the frequency content of the original raster.
However, no attempt is made at accurately increasing the high-frequency components of the raster, which are required by our definition for high-resolution geophysics.
Without sampling at a smaller separation or closer line spacing, it is still possible to increase the high-frequency content of aeromagnetic grids.

The rate at which the power of the component frequencies of the potential field diminish with increasing distance is well known.
Back-calculation can be made to predict the power of these frequencies at a smaller separation, and this is termed downward continuation \textcite{bullardDeterminationMassesNecessary1948,blakelyPotentialTheoryGravity1996}.
The inverse using a larger distance is termed upward continuation, and is useful for reducing potentially unwanted near-surface magnetic detail, or levelling data to a common datum \parencite[e.g.\ open file magnetic data in Australia is available levelled to AWAGS][]{mintyAirborneGeophysicalMapping2011}.
Downward continuation offers the benefit of high-resolution data from low-resolution sampling, and is therefore highly studied \parencite{zuoDownwardContinuationTransformation2020,liStableDownwardContinuation2023,fediStableDownwardContinuation2002,yeHighprecisionDownwardContinuation2022,zhangNumericalSolutionsMeanValue2018a,guoPotentialFieldContinuation2020,gangImprovedStableDownward2018,pilkingtonPotentialFieldContinuation2017}.
However, due to the inclusion of sensor noise, the practicable limit of numerical methods is six times the low-resolution cell size.
Recent approaches in machine learning that overcome this limit will be discussed in \Cref{sec:introml}, alongside alternative methods for enhancing aeromagnetic survey resolution.

\section{Machine learning for enhancing resolution}
Representation learning...


\label{sec:introml}
Deep learning is a field of contemporary machine learning, where additional levels of learning are undertaken between the input and output.
The current generation of deep learning today began in ernest in 2006 \parencite[as defined by][]{dengDeepLearningMethods2014,Goodfellow-et-al-2016}, and has pervaded many if not all industries and research fields.
The network architectures encountered in this manuscript include convolutional neural networks (CNN), multilayer perceptrons (MLP), and generative adversarial networks (GAN).

CNN methods have long been well suited for computer vision tasks such as image classification and, more recently, super-resolution.
In an image raster, neighbouring pixels define simple features, and these features combine in the wider image to define complex features.
CNNs excel in deep learning tasks for these data, due in part to the structure of the receptive field of the network.
The receptive field is formed from stacked layers of small convolutional kernels, typically with a size of \numproduct{3 x 3} cells for SR CNNs.
These low-level kernels learn simple functions, which are convolved with deeper layers to learn increasingly complex filters across a larger image extent, and widening the receptive field.
A review of deep learning CNN is provided in \textcite{goodfellowDeepLearning2016}.
% Once trained, the learnt functions are conditioned to transform low-resolution features into high-resolution features, using the neighbouring input values of the low-resolution grid in the context of filters learnt from the training data.

Recently, CNNs have been surpassed by coordinate multilayer perceptron (CMLP) neural networks for the task of SR \parencite{chenLearningContinuousImage2021}.
CMLP networks learn a signal as a function of its coordinates, and SR is performed by predicting pixel values on the continuum of coordinates in the domain of the learnt function, including at scales that were not seen during training.
Key to the performance of these networks are implicit functions, which parameterise a function learned from training dataset with a neural network analogue.
The network analogue can then be queried for information that fits the learnt function but was not included in the training data.

% GANs are an architecture making use of two competing neural networks.


% Introduce different types of networks and their applications in geoscience and adjacent areas

There has been active area of computer vision research in developing various deep learning network architectures (*refs) to enhance the resolution of images (super-resolution) based on previously seen patterns.
% * talk about different superresolution methods that have been effective for some applications with refs*, and introduce learning implicit representation from imagery and their application for improving resolution *** I think it's useful to link this resolution improvement in geophysics as imputation problem.


%  deep learning methods that consider the characteristics of geophysics potential fields data.



*needs a bit of literature review on what's been done before in this space in geoscience***
to the best of my knowledge, this use of super-resolution ***

\section{}
\label{sec:introgeoml}


The earliest application of CNN SR was demonstrated by \parencite{dongLearningDeepConvolutional2014}, where convolutional layers are used for low-resolution patch extraction, feature mapping, and reconstruction of high-resolution images.

The outputs of each convolutional layer are similar to hand-crafted filters or feature dictionaries in earlier example-based super-resolution methods \parencite{freemanExamplebasedSuperresolution2002}, but the learnt filters are more numerous and fully trainable with the deep learning approach.
Following the success of \parencite{dongLearningDeepConvolutional2014}, many iterative improvements have been made to CNN based SR\@.
These include SRCNN \parencite{dongImageSuperresolutionUsing2016}, RDN \parencite{zhangResidualDenseNetwork2018}, ESRGAN \parencite{wangESRGANEnhancedSuperresolution2018}, and others \parencite{ledigPhotorealisticSingleImage2017,limEnhancedDeepResidual2017}.
While each of these works contribute unique features to CNN-based super-resolution, they follow a common stacked block design.
This comprises a set of initial input feature extraction convolutions, a sequence of convolutional blocks with activations (neurons), some number and arrangement of intermediate residual learning pathways (skip connections), and an upsampling block for feature upscaling and convolving latent features back to image space.


- MLP SR
% One limitation of these CNN-based networks is the challenge of upscaling data at non-integer scales, and scales beyond those in the training dataset.
% This was achieved by \parencite{huMetaSRMagnificationarbitraryNetwork2019,wangLearningSingleNetwork2021}, however recent approaches using implicit function based networks have improved performance.

- Deep learning downward continuation


This thesis, explores the applications of
And the potential field has frequency based details to interpet sources at varying depth, etc...
how this is directly relevant to implicit neural representation learning....
This needs to set the scence of what is to come...

\section{Research Aims}
The overall aim of this thesis is to adapt and extend the use of deep learning to improve the enhancement and processing of geophysics data. The three aims of this thesis are as follows:

\begin{itemize}
    \item{} how different deep learning based super-resolution methods previously developed for natural imagery are applicable to magnetic potential fields grids that has different in dynamic value range in a single channel.

    \item{}Extending the understanding of effectiveness of a super-resolution technique in terms of resolution scale specifically in relation to survey line spacing; and the model construction framework to ensure the model is trained for diverse types of geological features manifested in magnetic data.

    \item{}Evaluating the use of implicit neural representation technique that performs representation learning of potential field survey point and line data, which can generate a grid directly from survey data by learning *** within data, and their analytical gradient calculation from the network.
\end{itemize}

The above aims are achieved using case studies using open access aeromagnetic data provided by Geoscience Australia.
Case studies are critical in promoting the adoption of applications in geophysics, by demonstrating a methods suitability to the data and task of interest.
Inference on real-world data is generally the aim of machine learning applications, however these data present many challenges \parencite{nikolenkoSyntheticDataDeep2021,tremblayTrainingDeepNetworks2018}.
Especially in the case of geophysics, these challenges include the paucity of ground truth, high dimensionality, and measurement noise and uncertainty \parencite{karpatneMachineLearningGeosciences2019}.


\section{Research Contributions and Thesis Structure}
This thesis is presented as a series of papers which are published or in-preparation for publication.
These are presented in Cref{ch:paper1,ch:paper2,ch:paper3}, each with a brief contextual introduction.
Key findings and contributions of the research are discussed in Cref{ch:discussion}, followed by overall thesis conclusions in Cref{ch:conclusion}.

Cref{ch:paper1} presents published work on an initial investigation into the technique of deep learning super-resolution for magnetic potential field data.
To our knowledge, this chapter presents the first investigation of deep learning super-resolution of gridded magnetic potential field data within the literature.
Two convolutional neural network models namely RDN and ESRGAN from the contemporary literature are implemented \parencite{zhangResidualDenseNetwork2018,limEnhancedDeepResidual2017}.
With uncertainty on the amount of data required to train a SR model for magnetic grids, a dataset comprising readily available high- and low-resolution TMI grid pairs is formed using published state magnetic compilation maps of Western Australia \parencite{brett20MagneticMerged2020}.
These data contain magnetic features and textures from a range of geological domains.
The chapter concludes by observing the success of SR on magnetic features, characterising the performance of the two adapted networks, and recommending the method based on RDN as being more reliable for the SR task.
The primary contribution of Cref{ch:paper1} is establishing the capacity of the SR networks developed for natural images in computer vision to extend to the distinctly different data of low-resolution aeromagnetic geophysics.

Following the findings of the previous chapter, Cref{ch:paper2} addresses the need for more generalisable model training for super-resolution.
%TODO refactor per paper 2 intro - primary contrbution is diverse geology with synthetci data. Don't say low-resolution data
It undertakes these aims by developing a resampling and regridding approach for genreating realistic low-resolutoin grids from real-world high-resolution state magnetic data and synthetic data from the contemporaneously published collection of magnetic grids forward modelled from realistic geology \parencite{jessellNoddyverseMassiveData2022}.
Furthermore, it adopts advances in the machine learning literature by implementing the LTE network \parencite{leeLocalTextureEstimator2022}.
The proposed low-resolution transform simulates aeromagnetic line sampling of a potential field extent at \qty{320}{\m} and \qty{80}{\m}, which are in the regional and prospect scale respectively.
Using the transformed synthetic data, a baseline model is trained on data from over \num{300} distinct geological histories.
While the baseline model can generalise to real-world data, some incorrect reconstructions remain.
By fine-tune training the model with transformed real-world state map data, these errors are rectified, and the structural accuracy is improved.

Cref{ch:paper3} reports the use of coordinate multilayer perceptron (CMLP) neural networks to transform line data into a grid that eliminating levelling and gridding
%TODO ensure refer back to aims
*** learn a representation of the function that describes a surveyed potential field extent.
These networks are highly suited for processing spatial data such as point sampled geophysics surveys.
Using the proposed method, regularly spaced grids are directly predicted from scattered survey point data.
Additionally, automatic differentiation is performed with the representation to calculate derivatives of a potential field.
Both tasks are fundamental to interpretation and processing in geophysics, and by commencing the methods from point sample data, the number of individual processing steps is reduced, and the application is fully trainable.
To our knowledge, this research is the first description of deep learning implicit neural representation for aeromagnetic potential field data in the literature.

Cref{ch:discussion} presents a synthesis discussion for the prior chapters. Given the recency of deep learning super-resolution and implicit neural representation, multiple avenues of future opportunities are identified.
These aim to increase the adoption of super-resolution in geophysics by addressing apprehension toward the reliability of interpretation of the outputs.
A synthesis comparison of the SR networks adapted in Cref{ch:paper1,ch:paper2} is also presented, which prompts questions regarding model and dataset construction.
The use of a priori information well understood to geophysics is also discussed, with potential application in both super-resolution and implicit neural representation.

Finally, overall conclusions are presented in Cref{ch:conclusion}.

\end{document}