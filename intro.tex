\documentclass[manuscript.tex]{subfiles}
\begin{document}


\section{Introduction} %  indicates that an original and coherent argument is being developed and that the research is making a contribution to existing literature
% outline the topic, justify the research, identify research aims and outline thesis structure
\subsection{Big geodata}
(2 pg)
%% Airborne surveys are routinely conducted as pre-competitive regional data acquisition to stimulate exploration, or as targeted investigations by the exploration industry.
%% Surveys are also routinely collected on the ground or by satellite, with processing methods and interpretation shared between all forms of acquisition.
%% The collected data may contain millions of discrete samples scattered in three dimensions,
%% Interpretation often relies on regularisation to a quantised grid, allowing digital processing and human or machine interpretation.
The subsurface continuation of exposed geology can be extrapolated and interpolated using structures in the magnetic responses.
For this reason, magnetic field survey grids remain a standard tool in the geological mapping and interpration toolbox. 
Open file data from government or publicly released commercial surveys form part of the greenfields exploration.
Case studies on open file data are presented throughout to develop the method.
%% Budgets: Flying, handling, storage,

% Enhance the information extractable from potential field surveys.
% In combination they reflect ongoing advances in deep learning and,
% their extension to challenges and opportunities in geophysics.

\subsection{Research Contributions and Thesis Structure}
% paper or series of papers suitable for publication in scholarly journals; or a combination of published and unpublished work
This thesis is presented as a series of papers which are published or in-preparation for publication in scholarly journals.
Chapter 1 presents a published manuscript, typefaced for consistency with the remaining thesis.
Chapters 2 and 3 present original research in preparation as articles for publication.
A discussion arising from these works is provided in chapter 4, and a brief overview of the content and contributions of each chapter is given here.

Chapters 2 and 3 

\end{document}