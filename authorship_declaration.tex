% \documentclass[manuscript.tex]{subfiles}
% \begin{document}

\addsec{Authorship Declaration}
This research was carried out at the University of Western Australia.
This thesis contains material that has been published or prepared for publication.
The extent of the candidate's contribution toward publication is outlined below.

\smallskip{}
\noindent{}All software development for the research was performed by L. Smith.

\noindent{}The possible application of super-resolution for geophysical grid data was suggested by E.J. Holden.

\medskip{}
\noindent{}Sections comprising published journal articles:
\begin{itemize}
      \item{}\textbf{Location in thesis}: Chapter 2

            \textbf{Publication}: Smith, L., Horrocks, T., Holden, E.J., Wedge, D., Akhtar, N., 2022. Magnetic grid resolution enhancement using machine learning: A case study from the Eastern Goldfields Superterrane. Ore Geology Reviews 150, 105119.
            \url{https://doi.org/10.1016/j.oregeorev.2022.105119}
            
            \textbf{Contribution}: The proportion of original work by the student is \qty{80}{\percent}.
            The method was researched, developed, analysed and drafted by L. Smith, who revised the manuscript with feedback from co-authors E.J. Holden, D. Wedge, T. Horrocks, and N. Akhtar.
            As a published manuscript, further revision was undertaken prompted by peer review.
\end{itemize}

\noindent{}Sections comprising manuscripts in preparation:
\begin{itemize}
      \item{}
            \textbf{Location in thesis}: Chapter 3

            \textbf{Contribution}: The proportion of original work by the student is \qty{80}{\percent}.
            Further investigation of super-resolution was initiated, developed, analysed, and prepared by L. Smith, with critical manuscript review from N. Akhtar, E.J. Holden, T. Horrocks, and D. Wedge.

      \item{}
            \textbf{Location in thesis}: Chapter 4

            \textbf{Contribution}: The proportion of original work by the student is \qty{90}{\percent}.
            The concept and development of implicit neural representation for geophysics was initiated by L. Smith, who performed the subsequent research, development, analysis, and writing. The manuscript was assisted by review from N. Akhtar, E.J. Holden, T. Horrocks, and D. Wedge.

\end{itemize}

\newpage{}
\noindent{}The contributions of the author have been truthfully stated, and permission from
each author has been sought to include the published material in this thesis.

% \vspace*{20 mm}

\noindent\begin{tabular}{ll}
                                            &                           \\[8ex]
      \makebox[100 mm]{\dotfill}            & \makebox[30 mm]{\dotfill} \\
      Eun-Jung Holden                       & Date                      \\
      Principal and Coordinating Supervisor &                           \\[8ex]
      \makebox[100 mm]{\dotfill}            & \makebox[30 mm]{\dotfill} \\
      Daniel Wedge                          & Date                      \\
      Co-supervisor                         &                           \\[8ex]
      \makebox[100 mm]{\dotfill}            & \makebox[30 mm]{\dotfill} \\
      Tom Horrocks                          & Date                      \\
      Co-supervisor                         &                           \\[8ex]
      \makebox[100 mm]{\dotfill}            & \makebox[30 mm]{\dotfill} \\
      Naveed Akhtar                         & Date                      \\
      Co-supervisor                         &                           \\
\end{tabular}

% \end{document}